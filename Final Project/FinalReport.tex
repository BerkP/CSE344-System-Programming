\documentclass[12pt]{report}
\usepackage{amsmath}
\usepackage{latexsym}
\usepackage{amsfonts}
\usepackage[normalem]{ulem}
\usepackage{soul}
\usepackage{array}
\usepackage{amssymb}
\usepackage{extarrows}
\usepackage{graphicx}
\usepackage{subfig}
\usepackage{wrapfig}
\usepackage{wasysym}
\usepackage{enumitem}
\usepackage{adjustbox}
\usepackage{ragged2e}
\usepackage[svgnames,table]{xcolor}
\usepackage{tikz}
\usepackage{longtable}
\usepackage{changepage}
\usepackage{setspace}
\usepackage{hhline}
\usepackage{multicol}
\usepackage{tabto}
\usepackage{float}
\usepackage{multirow}
\usepackage{makecell}
\usepackage{fancyhdr}
\usepackage[toc,page]{appendix}
\usepackage[hidelinks]{hyperref}
\usetikzlibrary{shapes.symbols,shapes.geometric,shadows,arrows.meta}
\tikzset{>={Latex[width=1.5mm,length=2mm]}}
\usepackage{flowchart}\usepackage[paperheight=11.69in,paperwidth=8.27in]{geometry}
\usepackage[utf8]{inputenc}
\usepackage[T1]{fontenc}
\TabPositions{0.49in,0.98in,1.47in,1.96in,2.45in,2.94in,3.43in,3.92in,4.41in,4.9in,5.39in,5.88in,}
\urlstyle{same}
\renewcommand{\_}{\kern-1.5pt\textunderscore\kern-1.5pt}
\setcounter{tocdepth}{5}
\setcounter{secnumdepth}{5}
\setlistdepth{9}
\renewlist{enumerate}{enumerate}{9}
		\setlist[enumerate,1]{label=\arabic*)}
		\setlist[enumerate,2]{label=\alph*)}
		\setlist[enumerate,3]{label=(\roman*)}
		\setlist[enumerate,4]{label=(\arabic*)}
		\setlist[enumerate,5]{label=(\Alph*)}
		\setlist[enumerate,6]{label=(\Roman*)}
		\setlist[enumerate,7]{label=\arabic*}
		\setlist[enumerate,8]{label=\alph*}
		\setlist[enumerate,9]{label=\roman*}

\renewlist{itemize}{itemize}{9}
		\setlist[itemize]{label=$\cdot$}
		\setlist[itemize,1]{label=\textbullet}
		\setlist[itemize,2]{label=$\circ$}
		\setlist[itemize,3]{label=$\ast$}
		\setlist[itemize,4]{label=$\dagger$}
		\setlist[itemize,5]{label=$\triangleright$}
		\setlist[itemize,6]{label=$\bigstar$}
		\setlist[itemize,7]{label=$\blacklozenge$}
		\setlist[itemize,8]{label=$\prime$}

\pagenumbering{gobble}
\setlength{\topsep}{0pt}\setlength{\parskip}{8.04pt}
\setlength{\parindent}{0pt}
\renewcommand{\arraystretch}{1.3}

\begin{document}
\begin{Center}
{\fontsize{16pt}{19.2pt}\selectfont \textbf{CSE344 – System Programming}}\\
{\fontsize{16pt}{19.2pt}\selectfont \textbf{Final Project – Report}}
\end{Center}
\begin{FlushLeft}
{\fontsize{14pt}{16.8pt}\selectfont Berk Pekgöz}\\
{\fontsize{14pt}{16.8pt}\selectfont 171044041}
\end{FlushLeft}
\begin{FlushLeft}
{\fontsize{18pt}{21.6pt}\selectfont \textbf{\uline{Solving The Problem $\&$  Design:}}}
\end{FlushLeft}
\begin{FlushLeft}
{\fontsize{16pt}{19.2pt}\selectfont \textbf{Server Part}}
\end{FlushLeft}
\begin{FlushLeft}
\textbf{On server process first executes these steps in order:}
\end{FlushLeft}
\begin{itemize}
	\item Taking measures against double instantiation
	\item Parsing command line arguments(with getopt)
	\item Becoming daemon
	\item Opening log file as writing
	\item Printing executing parameters
	\item Settings for SIGINT handling
	\item Opening dataset file for reading
	\item Creating Database and Filling Database
	\item Creating pool threads
	\item Initializing socket queue
	\item Socket, bind and listen
\end{itemize}

\vspace{\baselineskip}
\begin{FlushLeft}
\uline{Taking measures against double instantiation:}\\
To prevent double instantiation, I used named semaphore with O\_CREAT $ \vert $  O\_EXCL flags. With this way, if another server program try to create semaphore, returns error. Name of named semaphore is defined at top the .c file.
\end{FlushLeft}

\vspace{\baselineskip}
\begin{FlushLeft}
\uline{Becoming daemon:}\\
To make the server daemon, \textit{same steps with the example which is shown in lecture is coded}. There is a small difference that only one flag can be given to the function and that flag controls \textit{umask()} call.
\end{FlushLeft}

\vspace{\baselineskip}
\begin{FlushLeft}
\uline{Settings for SIGINT handling:}\\
To achieve interrupt signal handling, first a handler function is set with \textit{sigaction()}. Then, SIGINT is masked for threads with \textit{pthread\_sigmask() }function. After the pool threads are created, mask is recovered for only main thread. With this way, only main thread can catch interrup signal and synchronization of the program is making it possible to achieve handling signal and terminating.
\end{FlushLeft}

\vspace{\baselineskip}
\begin{FlushLeft}
\uline{Creating Database and Filling Database:}\\
After the dataset file opened for reading, program reads first row of the file and gets number of columns. Then read rest of the file line by line and gets number of columns. After these reads program allocates space for database with number of columns and number of rows. Then, read file again and fills allocated database with entries. Size of entry string(char array) is defined at top of the .c file. Database structures are defined at beginning\  of the code.
\end{FlushLeft}

\vspace{\baselineskip}
\begin{FlushLeft}
\uline{Socket, bind and listen:}\\
To connect to socket, these functions are called in order: \textit{socket(), setsockopt(), bind()} and \textit{listen()}.  Also \textit{htons(port)} function is called.
\end{FlushLeft}
\begin{FlushLeft}
\textbf{After the server initialization part and start barrier...}
\end{FlushLeft}

\vspace{\baselineskip}
\begin{FlushLeft}
\uline{Main thread cycle:}\\
This cycle is a while loop. First accepts a connection from socket and puts that socket to the socket queue ( This queue structure is same with queue structure in HW$\#$ 4). After that there is a synchronization which is very similar to producer consumer problem and main thread is the producer side of the problem. This synchronization is achieved with mutexes and condition variables. 
\end{FlushLeft}
\begin{FlushLeft}
Waits with $``$\textit{empty$"$  }condition variable until one of the pool threads signals that condition variable. And checks for busythreads and term\_flag for this condition variable.
\end{FlushLeft}
\begin{FlushLeft}
After getting lock of that problem increments \textit{busythreads }and \textit{socketcount}. These are shared data between main thread and pool threads and provide synchronization.
\end{FlushLeft}
\begin{FlushLeft}
Sends signal to $``$full$"$  condition variable of pool threads to make them unblocked and check.
\end{FlushLeft}
\begin{FlushLeft}
If there are no available thread to assign connection, thread will be blocked with condition variable $``$\textit{empty$"$ }.
\end{FlushLeft}

\vspace{\baselineskip}
\begin{FlushLeft}
\uline{Pool threads:}\\
Pool threads are executes a while loop. In that loop these steps are executed in order: wait until job assigned(this is the consumer part of the problem which is mentioned in main thread cycle part), get a socket file descriptor from socket queue (this queue is protected with mutex), parse query, get reader/writer lock, execute query and finally release reader/writer lock. A thread continues reading from same socket until receive $``$end$"$  from that socket.
\end{FlushLeft}
\begin{FlushLeft}
Consumer part of producer/consumer problem is achieved with $``$full$"$  condition variable. And that condition variable checks $``$socketcount$"$  and $``$term\_flag$"$ .
\end{FlushLeft}
\begin{FlushLeft}
After completing job with current connection, decreases $``$busythreads$"$  and signals to main thread’s $``$empty$"$  condition variable.
\end{FlushLeft}
\begin{FlushLeft}
\textit{Queries with DISTINCT keyword does not work as asked, they behave as normal SELECT query.}
\end{FlushLeft}

\vspace{\baselineskip}
\begin{FlushLeft}
\uline{Server side of connection:}\\
To communicate with clients, there is a communication protocol. To send data from server to client, first server sends a integer. If that integer is zero or less, it means server will send a message with BUFFER\_LEN. If that integer is greater than zero, it means server will send table information and that integer is the column size. After that server sends another integer which means row size of the table which will be sent. Then server starts sending column $\ast$  row times entries with ENTRY\_LEN. After that wait for another query or $``$end$"$  message.
\end{FlushLeft}

\vspace{\baselineskip}
\begin{FlushLeft}
\uline{Terminating:}\\
Program handles SIGINT with int\_handler() function. Only main thread can catch signal(explained above). After handling, term\_flag becomes true and threads can not continue their loops. After exiting loops, pool threads returns to main thread. When flag becomes TRUE, main thread’s loop can not continue too and exits loop. After exiting loop, main thread joins all pool threads and free all resources then exit.
\end{FlushLeft}

\vspace{\baselineskip}
\begin{FlushLeft}
{\fontsize{16pt}{19.2pt}\selectfont \textbf{Client Part}}
\end{FlushLeft}
\begin{FlushLeft}
\textbf{Client has simple work.}
\end{FlushLeft}
\begin{itemize}
	\item First, parses command line arguments with getopt().
	\item Open the file which contains queries.
	\item Open socket and connect.\\
Uses inet\_addr() function to convert ip address.
	\item Read lines from file until EOF.
	\item If line start with client’s own id, then sends that query to the server using connected socket and waits for server’s response.
	\item Receives column size and row size.
	\item Creates temporary table with given sizes. Size of entry string(char array) is defined at top of the .c file. Database structures are defined at beginning of the code.
	\item Fill that table with entries.(Exactly row $\ast$  column entries will be received).
	\item Print that table
	\item Continue reading from file.
	\item If EOF reached, send $``$end$"$  message with BUFFER\_LEN to server for informing.
	\item Exit program.
\end{itemize}

\vspace{\baselineskip}
\vspace{\baselineskip}
\begin{FlushLeft}
{\fontsize{16pt}{19.2pt}\selectfont \textbf{\uline{Notices:}}}
\end{FlushLeft}
\setlength{\parskip}{0.0pt}
\begin{itemize}
	\item In client output, print of entries formatted with $\%$ 20.20s to be nicely formatted.
	\item There is timestamp on each row for both client and server.
\setlength{\parskip}{8.04pt}
	\item Dataset file must be in csv structure. 
	\item Input files are not modified.
	\item Server is a daemon process and code of become\_daemon() is very similar to the code shown in lecture.
	\item Server process has no controlling terminal.
	\item Length of table entries defined at top of .c file as ENTRY\_LEN. You can change that value. 
	\item Length of buffers defined at top of .c file as BUFFER\_LEN. You can change that value.
	\item All kinds of synchronization between thread achieved with mutexes and condition variables. 
	\item Only one instance of server can be executing at the same time.
	\item Main thread clean up after all threads. 
	\item In case of an arbitrary error, exit by printing to stderr a nicely formatted informative message and free resources. 
	\item In case of SIGINT the program (that means all threads), return all resources to the system and exit with an information message. 
	\item If the required command line arguments are missing/invalid, program print usage information and exit. 
	\item All requirements mentioned in PDF except $``$DISTINCT$"$  keyword have been achieved.
	\item No memory leak with valgrind.
\end{itemize}
\vspace{\baselineskip}
\end{document}